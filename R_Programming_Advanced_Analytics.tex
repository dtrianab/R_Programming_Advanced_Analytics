\PassOptionsToPackage{unicode=true}{hyperref} % options for packages loaded elsewhere
\PassOptionsToPackage{hyphens}{url}
%
\documentclass[]{article}
\usepackage{lmodern}
\usepackage{amssymb,amsmath}
\usepackage{ifxetex,ifluatex}
\usepackage{fixltx2e} % provides \textsubscript
\ifnum 0\ifxetex 1\fi\ifluatex 1\fi=0 % if pdftex
  \usepackage[T1]{fontenc}
  \usepackage[utf8]{inputenc}
  \usepackage{textcomp} % provides euro and other symbols
\else % if luatex or xelatex
  \usepackage{unicode-math}
  \defaultfontfeatures{Ligatures=TeX,Scale=MatchLowercase}
\fi
% use upquote if available, for straight quotes in verbatim environments
\IfFileExists{upquote.sty}{\usepackage{upquote}}{}
% use microtype if available
\IfFileExists{microtype.sty}{%
\usepackage[]{microtype}
\UseMicrotypeSet[protrusion]{basicmath} % disable protrusion for tt fonts
}{}
\IfFileExists{parskip.sty}{%
\usepackage{parskip}
}{% else
\setlength{\parindent}{0pt}
\setlength{\parskip}{6pt plus 2pt minus 1pt}
}
\usepackage{hyperref}
\hypersetup{
            pdftitle={R Programming Advanced Analytics In R For Data Science},
            pdfborder={0 0 0},
            breaklinks=true}
\urlstyle{same}  % don't use monospace font for urls
\usepackage[margin=1in]{geometry}
\usepackage{color}
\usepackage{fancyvrb}
\newcommand{\VerbBar}{|}
\newcommand{\VERB}{\Verb[commandchars=\\\{\}]}
\DefineVerbatimEnvironment{Highlighting}{Verbatim}{commandchars=\\\{\}}
% Add ',fontsize=\small' for more characters per line
\usepackage{framed}
\definecolor{shadecolor}{RGB}{248,248,248}
\newenvironment{Shaded}{\begin{snugshade}}{\end{snugshade}}
\newcommand{\AlertTok}[1]{\textcolor[rgb]{0.94,0.16,0.16}{#1}}
\newcommand{\AnnotationTok}[1]{\textcolor[rgb]{0.56,0.35,0.01}{\textbf{\textit{#1}}}}
\newcommand{\AttributeTok}[1]{\textcolor[rgb]{0.77,0.63,0.00}{#1}}
\newcommand{\BaseNTok}[1]{\textcolor[rgb]{0.00,0.00,0.81}{#1}}
\newcommand{\BuiltInTok}[1]{#1}
\newcommand{\CharTok}[1]{\textcolor[rgb]{0.31,0.60,0.02}{#1}}
\newcommand{\CommentTok}[1]{\textcolor[rgb]{0.56,0.35,0.01}{\textit{#1}}}
\newcommand{\CommentVarTok}[1]{\textcolor[rgb]{0.56,0.35,0.01}{\textbf{\textit{#1}}}}
\newcommand{\ConstantTok}[1]{\textcolor[rgb]{0.00,0.00,0.00}{#1}}
\newcommand{\ControlFlowTok}[1]{\textcolor[rgb]{0.13,0.29,0.53}{\textbf{#1}}}
\newcommand{\DataTypeTok}[1]{\textcolor[rgb]{0.13,0.29,0.53}{#1}}
\newcommand{\DecValTok}[1]{\textcolor[rgb]{0.00,0.00,0.81}{#1}}
\newcommand{\DocumentationTok}[1]{\textcolor[rgb]{0.56,0.35,0.01}{\textbf{\textit{#1}}}}
\newcommand{\ErrorTok}[1]{\textcolor[rgb]{0.64,0.00,0.00}{\textbf{#1}}}
\newcommand{\ExtensionTok}[1]{#1}
\newcommand{\FloatTok}[1]{\textcolor[rgb]{0.00,0.00,0.81}{#1}}
\newcommand{\FunctionTok}[1]{\textcolor[rgb]{0.00,0.00,0.00}{#1}}
\newcommand{\ImportTok}[1]{#1}
\newcommand{\InformationTok}[1]{\textcolor[rgb]{0.56,0.35,0.01}{\textbf{\textit{#1}}}}
\newcommand{\KeywordTok}[1]{\textcolor[rgb]{0.13,0.29,0.53}{\textbf{#1}}}
\newcommand{\NormalTok}[1]{#1}
\newcommand{\OperatorTok}[1]{\textcolor[rgb]{0.81,0.36,0.00}{\textbf{#1}}}
\newcommand{\OtherTok}[1]{\textcolor[rgb]{0.56,0.35,0.01}{#1}}
\newcommand{\PreprocessorTok}[1]{\textcolor[rgb]{0.56,0.35,0.01}{\textit{#1}}}
\newcommand{\RegionMarkerTok}[1]{#1}
\newcommand{\SpecialCharTok}[1]{\textcolor[rgb]{0.00,0.00,0.00}{#1}}
\newcommand{\SpecialStringTok}[1]{\textcolor[rgb]{0.31,0.60,0.02}{#1}}
\newcommand{\StringTok}[1]{\textcolor[rgb]{0.31,0.60,0.02}{#1}}
\newcommand{\VariableTok}[1]{\textcolor[rgb]{0.00,0.00,0.00}{#1}}
\newcommand{\VerbatimStringTok}[1]{\textcolor[rgb]{0.31,0.60,0.02}{#1}}
\newcommand{\WarningTok}[1]{\textcolor[rgb]{0.56,0.35,0.01}{\textbf{\textit{#1}}}}
\usepackage{graphicx,grffile}
\makeatletter
\def\maxwidth{\ifdim\Gin@nat@width>\linewidth\linewidth\else\Gin@nat@width\fi}
\def\maxheight{\ifdim\Gin@nat@height>\textheight\textheight\else\Gin@nat@height\fi}
\makeatother
% Scale images if necessary, so that they will not overflow the page
% margins by default, and it is still possible to overwrite the defaults
% using explicit options in \includegraphics[width, height, ...]{}
\setkeys{Gin}{width=\maxwidth,height=\maxheight,keepaspectratio}
\setlength{\emergencystretch}{3em}  % prevent overfull lines
\providecommand{\tightlist}{%
  \setlength{\itemsep}{0pt}\setlength{\parskip}{0pt}}
\setcounter{secnumdepth}{0}
% Redefines (sub)paragraphs to behave more like sections
\ifx\paragraph\undefined\else
\let\oldparagraph\paragraph
\renewcommand{\paragraph}[1]{\oldparagraph{#1}\mbox{}}
\fi
\ifx\subparagraph\undefined\else
\let\oldsubparagraph\subparagraph
\renewcommand{\subparagraph}[1]{\oldsubparagraph{#1}\mbox{}}
\fi

% set default figure placement to htbp
\makeatletter
\def\fps@figure{htbp}
\makeatother


\title{R Programming Advanced Analytics In R For Data Science}
\author{}
\date{\vspace{-2.5em}}

\begin{document}
\maketitle

\begin{Shaded}
\begin{Highlighting}[]
\KeywordTok{library}\NormalTok{(usethis)}
\end{Highlighting}
\end{Shaded}

\begin{verbatim}
## Warning: package 'usethis' was built under R version 3.6.3
\end{verbatim}

\url{https://www.youtube.com/watch?v=rf9q91A7hqM}

\url{https://plantronics.udemy.com/course/r-analytics/learn/lecture/5187978?start=30\#overview}

This is an \href{http://rmarkdown.rstudio.com}{R Markdown} Notebook
created in order to perform the tasks and examples shown in the course
``R Programming Advanced Analytics In R For Data Science'' published in
\href{https://www.udemy.com/}{Udemy} at
\href{https://www.udemy.com/course/r-analytics/}{Course}. Feel free to
dive into this document and use any of it. Feedbacks are always
welcome!.

This is one of my very first contact with R, I decided to start over
here having some knowledge already about data science and some other
tools such as MatLab or Python.

\hypertarget{section-2-data-preparation}{%
\subsection{Section 2: Data
preparation}\label{section-2-data-preparation}}

The actual directory of a project depends upon each own folder
allocation locally, the below command may help to get the current
directory that your R environment actually has.

\begin{Shaded}
\begin{Highlighting}[]
\CommentTok{#Get the actual directory}
\NormalTok{dir<-}\KeywordTok{getwd}\NormalTok{()}
\NormalTok{dir}
\end{Highlighting}
\end{Shaded}

\begin{verbatim}
## [1] "C:/Users/DTriana/Google Drive/E-Learning/R/R_Programming_Advanced_Analytics"
\end{verbatim}

In case that a particular directory needs to be set, the command setwd()
can be used for that purpose. In windows it may need to be used forward
slash while pointing to an specific directory. For instance see
\href{https://stackoverflow.com/questions/17605563/efficiently-convert-backslash-to-forward-slash-in-r}{Back
Slash to Forward Slash}

\hypertarget{dataset}{%
\subsection{Dataset}\label{dataset}}

The data is a financial set proposed by Kirill Eremenko and his team at
the following link:
\href{https://www.superdatascience.com/pages/rcourse-advanced}{Link}.

The below command read the data and further outputs display some primary
information based on the imported data.

\begin{Shaded}
\begin{Highlighting}[]
\NormalTok{fin <-}\KeywordTok{read.csv}\NormalTok{(}\StringTok{"P3-Future-500-The-Dataset.csv"}\NormalTok{)}
\end{Highlighting}
\end{Shaded}

\begin{Shaded}
\begin{Highlighting}[]
\CommentTok{#Top six rows}
\KeywordTok{head}\NormalTok{(fin,}\DecValTok{6}\NormalTok{)}
\end{Highlighting}
\end{Shaded}

\begin{verbatim}
##   ID         Name    Industry Inception Employees State           City
## 1  1     Over-Hex    Software      2006        25    TN       Franklin
## 2  2    Unimattax IT Services      2009        36    PA Newtown Square
## 3  3     Greenfax      Retail      2012        NA    SC     Greenville
## 4  4    Blacklane IT Services      2011        66    CA         Orange
## 5  5     Yearflex    Software      2013        45    WI        Madison
## 6  6 Indigoplanet IT Services      2013        60    NJ      Manalapan
##       Revenue          Expenses   Profit Growth
## 1  $9,684,527 1,130,700 Dollars  8553827    19%
## 2 $14,016,543   804,035 Dollars 13212508    20%
## 3  $9,746,272 1,044,375 Dollars  8701897    16%
## 4 $15,359,369 4,631,808 Dollars 10727561    19%
## 5  $8,567,910 4,374,841 Dollars  4193069    19%
## 6 $12,805,452 4,626,275 Dollars  8179177    22%
\end{verbatim}

\begin{Shaded}
\begin{Highlighting}[]
\CommentTok{#Botton 6 rows}
\KeywordTok{tail}\NormalTok{(fin, }\DecValTok{6}\NormalTok{)}
\end{Highlighting}
\end{Shaded}

\begin{verbatim}
##      ID             Name           Industry Inception Employees State
## 495 495  Rawfishcomplete Financial Services      2012       124    CA
## 496 496 Buretteadmirable        IT Services      2009        93    ME
## 497 497 Inventtremendous       Construction      2009        24    MN
## 498 498   Overviewparrot             Retail      2011      7125    TX
## 499 499       Belaguerra        IT Services      2010       140    MI
## 500 500      Allpossible        IT Services      2011        24    CA
##            City     Revenue          Expenses   Profit Growth
## 495 Los Angeles $10,624,949 2,951,178 Dollars  7673771    22%
## 496    Portland $15,407,450 2,833,136 Dollars 12574314    25%
## 497    Woodbury  $9,144,857 4,755,995 Dollars  4388862    11%
## 498  Fort Worth $11,134,728 5,152,110 Dollars  5982618    12%
## 499        Troy $17,387,130 1,387,784 Dollars 15999346    23%
## 500 Los Angeles $11,949,706   689,161 Dollars 11260545    24%
\end{verbatim}

It is importat to consider that the categorical variables appear as
factors with levels. R splits the categorical inputs into numbers or
identifiers to be processed in that way, which is rather easier than
strings or others for R.

When R cannot recognize a numeric varible, then it is read as factor.

\begin{Shaded}
\begin{Highlighting}[]
\CommentTok{#Compactly display the internal structure}
\KeywordTok{str}\NormalTok{(fin)}
\end{Highlighting}
\end{Shaded}

\begin{verbatim}
## 'data.frame':    500 obs. of  11 variables:
##  $ ID       : int  1 2 3 4 5 6 7 8 9 10 ...
##  $ Name     : Factor w/ 500 levels "Abstractedchocolat",..: 297 451 168 40 485 199 435 339 242 395 ...
##  $ Industry : Factor w/ 8 levels "","Construction",..: 8 6 7 6 8 6 3 2 6 3 ...
##  $ Inception: int  2006 2009 2012 2011 2013 2013 2009 2013 2009 2010 ...
##  $ Employees: int  25 36 NA 66 45 60 116 73 55 25 ...
##  $ State    : Factor w/ 43 levels "","AL","AZ","CA",..: 37 34 36 4 42 28 23 30 4 9 ...
##  $ City     : Factor w/ 297 levels "Addison","Alexandria",..: 94 181 105 195 151 154 53 295 232 26 ...
##  $ Revenue  : Factor w/ 499 levels "","$1,614,585",..: 480 195 486 247 403 142 309 1 97 118 ...
##  $ Expenses : Factor w/ 498 levels "","1,026,548 Dollars",..: 7 486 4 249 228 248 58 1 403 496 ...
##  $ Profit   : int  8553827 13212508 8701897 10727561 4193069 8179177 3259485 NA 5274553 11412916 ...
##  $ Growth   : Factor w/ 33 levels "","-2%","-3%",..: 15 17 12 15 15 19 13 1 27 17 ...
\end{verbatim}

Sometimes it is useful to define variables into categorical inputs, this
can be performed with factor() function and reassgining it into the
dataset.

\begin{Shaded}
\begin{Highlighting}[]
\CommentTok{#Changing from non-factor to factor}
\NormalTok{fin}\OperatorTok{$}\NormalTok{ID <-}\KeywordTok{factor}\NormalTok{(fin}\OperatorTok{$}\NormalTok{ID)}
\NormalTok{fin}\OperatorTok{$}\NormalTok{Inception <-}\StringTok{ }\KeywordTok{factor}\NormalTok{(fin}\OperatorTok{$}\NormalTok{Inception)}
\KeywordTok{str}\NormalTok{(fin)}
\end{Highlighting}
\end{Shaded}

\begin{verbatim}
## 'data.frame':    500 obs. of  11 variables:
##  $ ID       : Factor w/ 500 levels "1","2","3","4",..: 1 2 3 4 5 6 7 8 9 10 ...
##  $ Name     : Factor w/ 500 levels "Abstractedchocolat",..: 297 451 168 40 485 199 435 339 242 395 ...
##  $ Industry : Factor w/ 8 levels "","Construction",..: 8 6 7 6 8 6 3 2 6 3 ...
##  $ Inception: Factor w/ 16 levels "1999","2000",..: 8 11 14 13 15 15 11 15 11 12 ...
##  $ Employees: int  25 36 NA 66 45 60 116 73 55 25 ...
##  $ State    : Factor w/ 43 levels "","AL","AZ","CA",..: 37 34 36 4 42 28 23 30 4 9 ...
##  $ City     : Factor w/ 297 levels "Addison","Alexandria",..: 94 181 105 195 151 154 53 295 232 26 ...
##  $ Revenue  : Factor w/ 499 levels "","$1,614,585",..: 480 195 486 247 403 142 309 1 97 118 ...
##  $ Expenses : Factor w/ 498 levels "","1,026,548 Dollars",..: 7 486 4 249 228 248 58 1 403 496 ...
##  $ Profit   : int  8553827 13212508 8701897 10727561 4193069 8179177 3259485 NA 5274553 11412916 ...
##  $ Growth   : Factor w/ 33 levels "","-2%","-3%",..: 15 17 12 15 15 19 13 1 27 17 ...
\end{verbatim}

\hypertarget{factor-variable-trap-fvt}{%
\section{Factor Variable Trap (FVT)}\label{factor-variable-trap-fvt}}

\begin{Shaded}
\begin{Highlighting}[]
\NormalTok{a <-}\StringTok{ }\KeywordTok{c}\NormalTok{(}\StringTok{"12"}\NormalTok{,}\StringTok{"13"}\NormalTok{,}\StringTok{"14"}\NormalTok{,}\StringTok{"12"}\NormalTok{,}\StringTok{"12"}\NormalTok{)}
\NormalTok{a}
\end{Highlighting}
\end{Shaded}

\begin{verbatim}
## [1] "12" "13" "14" "12" "12"
\end{verbatim}

\begin{Shaded}
\begin{Highlighting}[]
\KeywordTok{typeof}\NormalTok{(a)}
\end{Highlighting}
\end{Shaded}

\begin{verbatim}
## [1] "character"
\end{verbatim}

\begin{Shaded}
\begin{Highlighting}[]
\NormalTok{b <-}\StringTok{ }\KeywordTok{as.numeric}\NormalTok{(a)}
\NormalTok{b}
\end{Highlighting}
\end{Shaded}

\begin{verbatim}
## [1] 12 13 14 12 12
\end{verbatim}

\begin{Shaded}
\begin{Highlighting}[]
\KeywordTok{typeof}\NormalTok{(b)}
\end{Highlighting}
\end{Shaded}

\begin{verbatim}
## [1] "double"
\end{verbatim}

\begin{Shaded}
\begin{Highlighting}[]
\NormalTok{z<-}\KeywordTok{factor}\NormalTok{(a)}
\NormalTok{z}
\end{Highlighting}
\end{Shaded}

\begin{verbatim}
## [1] 12 13 14 12 12
## Levels: 12 13 14
\end{verbatim}

\begin{Shaded}
\begin{Highlighting}[]
\NormalTok{y <-}\StringTok{ }\KeywordTok{as.numeric}\NormalTok{(z)}
\NormalTok{y}
\end{Highlighting}
\end{Shaded}

\begin{verbatim}
## [1] 1 2 3 1 1
\end{verbatim}

\end{document}
